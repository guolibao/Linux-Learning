\documentclass[paper=a4,oneside,fontsize=12pt,
  parskip=full]{scrartcl}
\usepackage[sc,osf]{mathpazo}
\usepackage[T1,small,euler-digits]{eulervm}
\usepackage[scaled=0.86]{berasans}
\usepackage[scaled=0.84]{beramono}
\begin{document}
\tableofcontents
\addsec{Introduction}
This document will be our starting point for simple
documents. It is suitable for a single page or up to
a couple of dozen pages.

The text will be divided into sections.
\section{The first section}
This first text will contain
\begin{itemize}
\item a table of contents,
\item a bulleted list,
\item headings and some text and math in section,
\item referencing such as to section \ref{sec:maths} and
      equation (\ref{eq:integral}).
\end{itemize}
We can use this document as a template for filling in
our own content.
\section{Some maths}
\label{sec:maths}
When we write a scientific or technical document, we usually
include math formulas. To get a brief glimpse of the look of
maths, we will look at an integral approximation of a function
$f(x)$ as a sum with weights $w_i$:
\begin{equation}
  \label{eq:integral}
  \int_a^b f(x)\,\mathrm{d}x \approx (b-a)
  \sum_{i=0}^n w_i f(x_i)
\end{equation}
\end{document}
