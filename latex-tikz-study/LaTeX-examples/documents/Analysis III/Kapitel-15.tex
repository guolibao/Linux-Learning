In diesem Kapitel sei \(\emptyset\neq B\subseteq\mdr^{2}\), \(B\)
kompakt, \(D\subseteq\mdr^{2}\) offen, \(B\subseteq D\)
und \(\varphi=(\varphi_{1},\varphi_{2},\varphi_{3})\in C^{1}(D,\mdr^{3})\). Das heißt: \(\varphi_{|B}\) ist eine Fläche mit
Parameterbereich \(B\), \(S:=\varphi(B)\)

\begin{definition}
\index{Oberflächenintegral}
Definiere die folgenden \textbf{Oberflächenintegrale}:
\begin{enumerate}
\item Sei \(f:\,S\to\mdr\) stetig. Dann:
\[
\int_{\varphi}{f\mathrm{d}\sigma}:=\int_{B}{f(\varphi(u,v))\lVert N(u,v)\rVert\mathrm{d}(u,v)}
\]
\item Sei \(F:\,S\to\mdr^{3}\) stetig. Dann:
\[
\int_{\varphi}{F\cdot n\mathrm{d}\sigma}:=\int_{B}{F(\varphi(u,v))\cdot N(u,v)\mathrm{d}(u,v)}
\]
\end{enumerate}
\end{definition}

\begin{beispiel}
Seien \(D,\,B,\,f,\,\varphi\) wie im letzten Beispiel in Kapitel 14.

Sei \(F(x,y,z):=(x,y,z)\); bekannt: \(N(u,v)=(-2u,-2v,1)\). Dann:
\begin{align*}
F(\varphi(u,v))\cdot N(u,v)&=F(u,v,u^{2}+v^{2})\cdot(-2u,-2v,1)\\
&=(u,v,u^{2}+v^{2})\cdot (-2u,-2v,1)\\
&=-(u^{2}+v^{2})
\end{align*}

Also:
\[
\int_{\varphi}{F\cdot n\mathrm{d}\sigma}=-\int_{B}{(u^{2}+v^{2})\mathrm{d}(u,v)}=-\frac{\pi}{2}
\]
\end{beispiel}

\begin{satz}[Integralsatz von Stokes]
\label{Satz 15.1}
Es sei \(B\) zulässig, \(\partial B=\Gamma_{\gamma}\), wobei \(\gamma=(\gamma_{1},\gamma_{2})\) wie zu Beginn des Kapitels
13 ist. Es sei \(\varphi\in C^{2}(D,\mdr^{3})\). Weiter sei \(G\subseteq\mdr^{3}\) offen, \(S\subseteq G\) und \(F=(F_{1},F_{2},F_{3})\in C^{1}(G,\mdr^{3})\). Dann:
\[
\underbrace{\int_{\varphi}{\rot F\cdot n\mathrm{d}\sigma}}_{\text{Oberflächenint.}}=
    \underbrace{\int_{\varphi\circ\gamma}{F(x,y,z)\cdot\mathrm{d}(x,y,z)}}_{\text{Wegint.}}
\]
\end{satz}

\begin{beispiel}
\(D,\,B,\,f,\,F\) und \(\varphi\) seien wie in obigem Beispiel.
% Bild einfuegen
Hier: \(\gamma(t)=(\cos t,\sin t)\quad(t\in [0,2\pi])\).
Dann: \((\varphi\circ\gamma)(t)=\varphi(\cos t, \sin t)=(\cos t, \sin t, 1)\quad(t\in [0,2\pi])\).

Es ist \(\rot F=0\), also: \(\int_{\varphi}{\rot F\cdot n\mathrm{d}\sigma}=0\)
\begin{align*}
\int_{\varphi\circ\gamma}{F(x,y,z)\mathrm{d}(x,y,z)}&=
    \int_{0}^{2\pi}{F((\varphi\circ\gamma)(t))\cdot(\varphi\circ\gamma)'(t)\mathrm{d}t}\\
&=\int_{0}^{2\pi}{F(\cos t,\sin t, 1)\cdot (-\sin t,\cos t,0)\mathrm{d}t}\\
&=\int_{0}^{2\pi}{\underbrace{(\cos t,\sin t,1)\cdot (-\sin t,\cos t,0)}_{=0}\mathrm{d}t}\\
&=0
\end{align*}
\end{beispiel}

\begin{beweis}
Sei \(\varphi:=\varphi\circ\gamma,\,\varphi=(\varphi_{1},\varphi_{2},\varphi_{3})\), also
    \(\varphi_{j}=\varphi_{j}\circ\gamma\quad(j=1,2,3)\).

Zu zeigen:
\begin{align*}
\int_{\varphi}{\rot F\cdot n\mathrm{d}\sigma}
    &=\int_{\varphi}{F(x,y,z)\mathrm{d}(x,y,z)}\\
    &=\int_{0}^{2\pi}{F(\varphi(t))\cdot\varphi'(t)\mathrm{d}t}\\
    &=\int_{0}^{2\pi}{\left(\sum_{j=1}^{3}{F_{j}(\varphi(t))\varphi_{j}'(t)}\right)\mathrm{d}t}\\
    &=\sum_{j=1}^{3}{\int_{0}^{2\pi}{F_{j}(\varphi(t))\varphi_{j}'(t)\mathrm{d}t}}
\end{align*}

Es ist \(\int_{\varphi}{\rot F\cdot n\mathrm{d}\sigma}=\int_{B}{\underbrace{(\rot F)(\varphi(x,y))\cdot(\varphi_{x}(x,y)\times\varphi_{y}(x,y))}_{=:g(x,y)}\mathrm{d}(x,y)}\).
Für \(j=1,2,3\):
\[
h_{j}(x,y):=\left(\underbrace{F_{j}(\varphi(x,y))\frac{\partial\varphi_{j}}{\partial y}(x,y)}_{=:u_{j}(x,y)},\underbrace{-F_{j}(\varphi(x,y))\frac{\partial\varphi_{j}}{\partial x}(x,y)}_{=:v_{j}(x,y)}\right)\quad((x,y)\in D)
\]


\(h_{j}=(u_{j},v_{j});\quad F\in C^{1},\,\varphi\in C^{2}\), damit folgt: \(h_{j}\in C^{1}\)

Nachrechnen: \(g=\mathrm{div} h_{1}+\mathrm{div} h_{2}+\mathrm{div} h_{3}\)

Damit:
\begin{align*}
\int_{B}{\rot F\cdot n\mathrm{d}\sigma}
    &=\sum_{j=1}^{3}{\int_{B}{\mathrm{div}\,h_{j}(x,y)\mathrm{d}(x,y)}}\\
    &=\sum_{j=1}^{3}{\int_{\gamma}{(u_{j}\mathrm{d}y-v_{j}\mathrm{d}x)}}\\
    &=\int_{0}^{2\pi}{F_{j}(\varphi(t))\varphi_{j}'(t)\mathrm{d}t}
\end{align*}
\end{beweis}
