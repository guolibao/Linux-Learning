\documentclass{article}

\usepackage[utf8]{inputenc} % this is needed for umlauts
\usepackage[ngerman]{babel} % this is needed for umlauts
\usepackage[T1]{fontenc}    % this is needed for correct output of umlauts in pdf

\usepackage[pdftex,active,tightpage]{preview}
\setlength\PreviewBorder{2mm}
\usepackage{tikz}
\usetikzlibrary{shapes,decorations,calc,patterns}
\usepackage{amsmath,amssymb}
\begin{document}
\begin{preview}
\begin{tikzpicture}[%
    auto,
    example/.style={
      rectangle,
      draw=blue,
      thick,
      fill=blue!20,
      text width=4.5em,
      align=center,
      rounded corners,
      minimum height=2em
    },
    algebraicName/.style={
      text width=7em,
      align=center,
      minimum height=2em
    },
    explanation/.style={
      text width=10em,
      align=left,
      minimum height=3em
    }
  ]
\pgfdeclarepatternformonly{north east lines wide}%
   {\pgfqpoint{-1pt}{-1pt}}%
   {\pgfqpoint{10pt}{10pt}}%
   {\pgfqpoint{9pt}{9pt}}%
   {
     \pgfsetlinewidth{3pt}
     \pgfpathmoveto{\pgfqpoint{0pt}{0pt}}
     \pgfpathlineto{\pgfqpoint{9.1pt}{9.1pt}}
     \pgfusepath{stroke}
    }


    % Big background
    \draw[fill=lime!20,lime!20, rounded corners]     (-1.8, 0.60) rectangle (10,-5);
    \draw[fill=purple!20,purple!20, rounded corners] (0.65, -3.15) rectangle (3.35,-3.85);
    \draw[fill=purple!20,purple!20, rounded corners] (4.65, -3.15) rectangle (7.35,-3.85);

    \draw[fill=blue!20,blue!20, rounded corners] (-1.35,-1.35) rectangle (1.35,-0.65);
    \draw[fill=blue!20,blue!20, rounded corners] (2.65,-1.35) rectangle (5.35,-0.65);
    \draw[fill=blue!20,blue!20, rounded corners] (6.65,-1.35) rectangle (9.35,-0.65);

    \draw (2, 0) node[algebraicName] (A) {Modul: Programmieren}
          (6, 0) node[explanation]   (X) {
            \begin{minipage}{0.9\textwidth}
                \tiny
                \begin{itemize}
                    \item 5 ECTS
                \end{itemize}
            \end{minipage}
          }
          (0,-1) node[algebraicName] (B) {Tutorium}
          (4,-1) node[algebraicName] (C) {Übung}
          (8,-1) node[algebraicName] (D) {Vorlesung}
          (0,-2) node[algebraicName] (E) {Student}
          (4,-2) node[algebraicName] (F) {Mitarbeiter}
          (8,-2) node[algebraicName] (G) {Dozent}
          (2,-3.5) node[algebraicName, purple] (H) {Übungsschein}
          (1.8,-4.35) node[explanation]   (X) {
            \begin{minipage}{\textwidth}
                \tiny
                \begin{itemize} \itemsep-0.4em
                    \item Muss bestanden werden
                    \item Keine Note
                    \item keine Bonuspunkte
                \end{itemize}
            \end{minipage}
          }
          (6,-3.5) node[algebraicName, purple] (I) {Klausur}
          (5.8,-4.3) node[explanation]   (X) {
            \begin{minipage}{\textwidth}
                \tiny
                \begin{itemize} \itemsep-0.4em
                    \item Muss bestanden werden
                    \item Abschlussnote ergibt Modulnote
                \end{itemize}
            \end{minipage}
          };

    \draw[blue, thick, rounded corners] ($(B.north west)$) rectangle ($(B.south east)$);
    \draw[blue, thick, rounded corners] ($(C.north west)$) rectangle ($(C.south east)$);
    \draw[blue, thick, rounded corners] ($(D.north west)$) rectangle ($(D.south east)$);

    \draw[purple, thick, rounded corners] ($(H.north west)$) rectangle ($(H.south east)$);
    \draw[purple, thick, rounded corners] ($(I.north west)$) rectangle ($(I.south east)$);

    \draw[lime, thick, rounded corners]   ($(B.north west)+(-0.1,0.1)$) rectangle ($(E.south east)+(0.1,-0.1)$);
    \draw[lime, thick, rounded corners]   ($(C.north west)+(-0.1,0.1)$) rectangle ($(F.south east)+(0.1,-0.1)$);
    \draw[lime, thick, rounded corners]   ($(D.north west)+(-0.1,0.1)$) rectangle ($(G.south east)+(0.1,-0.1)$);
\end{tikzpicture}
\end{preview}
\end{document}
