%!TEX root = booka4.tex

\section{Einleitung}
Kognitive Automobile sind, im Gegensatz zu klassischen Automobilen, in der Lage
ihre Umwelt und sich selbst wahrzunehmen und dem Fahrer zu assistieren oder
auch teil- bzw. vollautonom zu fahren. Diese Systeme benötigen Zugriff auf
Sensoren und Aktoren, um ihre Aufgabe zu erfüllen. So benötigt ein Auto mit
Antiblockiersystem beispielsweise die Drehzahl an jedem Reifen und die
Möglichkeit die Bremsen zu beeinflussen; für Einparkhilfen werden Sensoren
benötigt, welche die Distanz zu Hindernissen wahrnehmen sowie Aktoren, die das
Auto lenken und beschleunigen können. Weitere dieser Systeme sind
Spurhalteassistenz, Spurwechselassistenz und Fernlichtassistenz.

Diese Veröffentlichung ist wie folgt gegliedert: \Cref{ch:standards} geht auf
Standards wie den CAN-Bus und Verordnungen, die in der Europäischen Union
gültig sind, ein. In \cref{ch:attack} werden Angriffsziele und Grundlagen zu
den Angriffen erklärt, sodass in
\cref{ch:defense} mögliche Verteidigungsmaßnahmen erläutert werden können.
