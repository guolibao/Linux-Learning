\subsection{How can we use this massive amout of information?}
\begin{frame}{How can we use this massive amout of information?}
    \begin{itemize}[<+->]
        \item 625.3 million websites
        \item Wikipedia is one website and has several millions of pages
        \item[$\Rightarrow$] we need to rank websites!
    \end{itemize}
\end{frame}

\subsection{Idea}
\begin{frame}{Basics of PageRank}
    We all know that:
    \begin{itemize}[<+->]
        \item humans know what is good for them
        \item[\xmark] machines don't know what's good for humans
        \item humans create websites
        \item humans will only \href{http://en.wikipedia.org/wiki/Hyperlink}{link} to websites they like
        \item[$\Rightarrow$] hyperlinks are a quality indicator
    \end{itemize}
\end{frame}

\begin{frame}{How Could We Use That?}
    \begin{itemize}[<+->]
        \item simply count number of links to a website
        \item[\xmark] 10,000 links from only one page
        \item count number of websites that link to a website
        \item[\xmark] quality of the linking website matters
        \item[\xmark] total number of links on the source page matters
    \end{itemize}
\end{frame}

\framedgraphic{A Brilliant Idea}{../images/BrinPage.jpg}

\begin{frame}{Ideas of PageRank}
    \begin{itemize}[<+->]
        \item decisions of humans are complicated
        \item a lot of webpages get visited
        \item[$\Rightarrow$] modellize clicks on links as random behaviour
        \item links are important
        \begin{itemize}
            \item links of page A get less important, if A has many links
            \item links of page A get more important, if many link to A
        \end{itemize}
        \item[$\Rightarrow$] if B has a link from A, the rank of B increases by $\frac{Rank(A)}{Links(A)}$
    \end{itemize}

    \pause[\thebeamerpauses]

    \begin{algorithmic}
        \If{A links to B}
            \State $Rank(B)$ += $\frac{Rank(A)}{Links(A)}$
        \EndIf
    \end{algorithmic}
\end{frame}

\begin{frame}{What is PageRank?}
    The PageRank algorithm calculates the probability of a randomly
    clicking user ending up on a given page.
\end{frame}

\begin{frame}
    \begin{figure}
        \begin{tikzpicture}[->,scale=1.8, auto,swap]
            % Draw the vertices. First you define a list.
            \foreach \pos/\name in {{(1,0)/a}, {(4,0)/b}, {(5,1)/c},
                                    {(3,1)/d}, {(4,2)/e}, {(5,2)/f},
                                    {(1,2)/g}, {(0,3)/h}, {(2,4)/i},
                                    {(4,3)/j}, {(4,4)/k}, {(5,4)/l}}
                \node[vertex] (\name) at \pos {$\name$};

            % Connect vertices with edges and draw weights
            \foreach \source/ \dest /\pos in {g/a/, b/c/bend right,
                c/b/bend right, d/e/bend right, e/d/, h/g/, g/i/,g/i/bend left, g/i/bend right,
                h/i/, h/i/bend left, g/j/bend right, j/g/,
                j/k/bend right, k/j/bend right, j/f/, f/l/,e/j/,
                l/l/loop right, j/d/}
                \path (\source) edge [\pos] node {} (\dest);
        \end{tikzpicture}
    \end{figure}
\end{frame}

\pgfdeclarelayer{background}
\pgfsetlayers{background,main}

\begin{frame}
    \begin{figure}
        \begin{tikzpicture}[->,scale=1.8, auto,swap]
            % Draw the vertices. First you define a list.
            \foreach \pos/\name in {{(1,0)/a}, {(4,0)/b}, {(5,1)/c},
                                    {(3,1)/d}, {(4,2)/e}, {(5,2)/f},
                                    {(1,2)/g}, {(0,3)/h}, {(2,4)/i},
                                    {(4,3)/j}, {(4,4)/k}, {(5,4)/l}}
                \node[vertex] (\name) at \pos {$15$};

            % Connect vertices with edges and draw weights
            \foreach \source/ \dest /\pos in {g/a/, b/c/bend right,
                c/b/bend right, d/e/bend right, e/d/, h/g/, g/i/,g/i/bend left, g/i/bend right,
                h/i/, h/i/bend left, g/j/bend right, j/g/,
                j/k/bend right, k/j/bend right, j/f/, f/l/,e/j/,
                l/l/loop right, j/d/}
                \path (\source) edge [\pos] node {} (\dest);

             \node<3->[vertex] (a) at (1,0) {$17$};
             \node<5->[vertex] (g) at (1,2) {$21$};
             \node<9->[vertex] (i) at (2,4) {$19$};

            \begin{pgfonlayer}{background}
                \path<2->[selected edge] (g.center) edge [] node {2} (a.center);
                \path<4->[selected edge] (j.center) edge [] node {6} (g.center);
                \path<6->[selected edge] (g.center) edge [bend left] node {1} (i.center);
                \path<7->[selected edge] (g.center) edge [] node {0} (i.center);
                \path<8->[selected edge] (g.center) edge [bend right] node {3} (i.center);
            \end{pgfonlayer}


        \end{tikzpicture}
    \end{figure}
\end{frame}


%\begin{frame}{Ants}
%    \begin{itemize}[<+->]
%        \item Websites = nodes = anthill
%        \item Links = edges = paths
%        \item You place ants on each node
%        \item They walk over the paths
%        \item[] (at random, they are ants!)
%        \item After some time, some anthills will have more ants than
%              others
%        \item Those hills are more attractive than others
%        \item \# ants is probability that a random user would end on
%              a website
%    \end{itemize}
%\end{frame}

\begin{frame}{Mathematics}
    Let $x$ be a web page. Then
    \begin{itemize}
        \item $L(x)$ is the set of websites that link to $x$
        \item $C(y)$ is the out-degree of page $y$
        \item $\alpha$ is probability of random jump
        \item $N$ is the total number of websites
    \end{itemize}

    \[\displaystyle PR(x) := \alpha \left ( \frac{1}{N} \right ) + (1-\alpha) \sum_{y\in L(x)} \frac{PR(y)}{C(y)}\]
\end{frame}

\begin{frame}{Pseudocode}
        \begin{algorithmic}
\alertline<1>             \Function{PageRank}{Graph $web$, double $q=0.15$, int $iterations$} %q is a damping factor
%\alertline<2>                 \ForAll{$page \in web$}
%\alertline<3>                     \State $page.pageRank = \frac{1}{|web|}$ \Comment{intial probability}
%\alertline<2>                 \EndFor

\alertline<2>                 \While{$iterations > 0$}
\alertline<3>                     \ForAll{$page \in web$} \Comment{calculate pageRank of $page$}
\alertline<4>                         \State $page.pageRank = q$
\alertline<5>                         \ForAll{$y \in L(page)$}
\alertline<6>                             \State $page.pageRank$ += $\frac{y.pageRank}{C(y)}$
\alertline<5>                         \EndFor
\alertline<3>                     \EndFor
\alertline<2>                     \State $iterations$ -= $1$
\alertline<2>                 \EndWhile
\alertline<1>             \EndFunction
        \end{algorithmic}
\end{frame}
