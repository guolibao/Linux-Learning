\subsection{Spezielle Graphen}
\begin{frame}{Vollständige Graphen}
\begin{block}{Vollständiger Graph}
Sei $G = (E, K)$ ein Graph.

$G$ heißt \textbf{vollständig} $:\Leftrightarrow K = E \times E \setminus \Set{\Set{e, e} | e \in E}$
\end{block}

Ein vollständiger Graph mit $n$ Ecken wird als $K_n$ bezeichnet.
\pause
\tikzstyle{vertex}=[draw,fill=black,circle,minimum size=10pt,inner sep=0pt]
\begin{gallery}
    \galleryimage[Green]{vollstaendig/k-1}
    \galleryimage[Green]{vollstaendig/k-2}
    \galleryimage[Green]{vollstaendig/k-3}
    \galleryimage[Green]{vollstaendig/k-4}\\
    \galleryimage[Green]{vollstaendig/k-5}
    \galleryimage[Green]{vollstaendig/k-6}
    \galleryimage[Green]{vollstaendig/k-7}
    \galleryimage[Green]{vollstaendig/k-16}
\end{gallery}
\end{frame}

\begin{frame}{Bipartiter Graph}
\begin{block}{Bipartiter Graph}
Sei $G = (E, K)$ ein Graph und $A, B \subset E$ zwei disjunkte Eckenmengen mit
$E \setminus A = B$.

$G$ heißt \textbf{bipartit} $:\Leftrightarrow \forall_{k = \Set{e_1, e_2} \in K}: (e_1 \in A \text{ und } e_2 \in B) \text{ oder } (e_1 \in B \text{ und } e_2 \in A) $
\end{block}

\begin{gallery}
    \galleryimage[Green]{bipartit/k-2-2}
    \galleryimage[Green]{bipartit/k-2-3}
    \galleryimage[Green]{vollstaendig-bipartit/k-2-5}
    \galleryimage[Green]{vollstaendig-bipartit/k-3-3}\\
    \galleryimage[Green]{vollstaendig-bipartit/k-3-4}
    \galleryimage[Green]{vollstaendig-bipartit/k-3-5}
    \galleryimage[Green]{vollstaendig-bipartit/k-4-5}
    \galleryimage[Green]{vollstaendig-bipartit/k-5-5}
\end{gallery}
\end{frame}

\begin{frame}{Vollständig bipartiter Graph}
\begin{block}{Vollständig bipartiter Graph}
Sei $G = (E, K)$ ein bipartiter Graph und $\Set{A, B}$ bezeichne die Bipartition.

$G$ heißt \textbf{vollständig bipartit} $:\Leftrightarrow A \times B = K$
\end{block}

\begin{gallery}
    \galleryimage[red]{bipartit/k-2-2}
    \galleryimage[red]{bipartit/k-2-3}
    \galleryimage[Green]{vollstaendig-bipartit/k-2-5}
    \galleryimage[Green]{vollstaendig-bipartit/k-3-3}\\
    \galleryimage[Green]{vollstaendig-bipartit/k-3-4}
    \galleryimage[Green]{vollstaendig-bipartit/k-3-5}
    \galleryimage[Green]{vollstaendig-bipartit/k-4-5}
    \galleryimage[Green]{vollstaendig-bipartit/k-5-5}
\end{gallery}
\end{frame}

\begin{frame}{Vollständig bipartite Graphen}
Bezeichnung: Vollständig bipartite Graphen mit der Bipartition $\Set{A, B}$
bezeichnet man mit $K_{|A|, |B|}$.

\begin{gallery}
    \galleryimage[Green]{vollstaendig-bipartit/k-2-2}
    \galleryimage[Green]{vollstaendig-bipartit/k-2-3}
    \galleryimage[Green]{vollstaendig-bipartit/k-2-5}
    \galleryimage[Green]{vollstaendig-bipartit/k-3-3}\\
    \galleryimage[Green]{vollstaendig-bipartit/k-3-4}
    \galleryimage[Green]{vollstaendig-bipartit/k-3-5}
    \galleryimage[Green]{vollstaendig-bipartit/k-4-5}
    \galleryimage[Green]{vollstaendig-bipartit/k-5-5}
\end{gallery}
\end{frame}

\begin{frame}{Aufgabe 2}
Wie viele Ecken und wie viele Kanten hat der $K_{m, n}$?

\visible<2>{
    \begin{align}
        \text{Ecken: }  &m+n\\
        \text{Kanten: } &m\cdot n
    \end{align}
}
\end{frame}
