% Martins Content
\subsection{Einleitung - Wie beginnen?}
\begin{frame}{Einleitung - Wie beginnen?}
    \begin{itemize}[<+->]
        \item Allgemeiner Vorspann
        \item Einstieg ins Thema
        \item Dient als Anwärmphase: Zuhörer \dots
           \begin{itemize}[<+->]
            \item \dots schließen vorangegangene Gedanken ab
            \item \dots stellen sich auf das Thema ein
            \item \dots schalten auf "`Empfang"'
           \end{itemize}
    \end{itemize}

    \pause[\thebeamerpauses]
    \begin{alertblock}{Wichtig}
        Der Einstieg muss zum Thema, der Redesituation und zu dir
        passen!
    \end{alertblock}
\end{frame}

\subsection{Situationen der Einleitung}
\begin{frame}{Situationen der Einleitung}
    \begin{itemize}[<+->]
        \item Anrede mit Blickkontakt
        \item Eigene Vorstellung des Präsentierenden (bei uns wohl eher nicht)
        \item Benennen des Themas
        \item Eigene Beweggründe
        \item "`Heitere"' Einstiegsbemerkung, These, Anekdote o.ä.
        \item Inhaltliche Übersicht
        \item Fragemodus klären
        \item Ggf. Oranisatorisches, z.B. Zeitplan, Raucherzonen, Getränke
    \end{itemize}

    \pause[\thebeamerpauses]
    \begin{block}{Info}
        In der Einleitung kommt es darauf an, mit dem ersten Kontakt
        die Aufmerksamkeit, das Interesse und das Vertrauen der
        Zuhörer zu gewinnen.
    \end{block}
\end{frame}

\begin{frame}{Bestandteile der Einleitung}
    Jeweils höchstens 2-3 Sätze zu \dots
    \begin{itemize}[<+->]
        \item \dots "`Startsignal"' ("`Auftritt"', Stimme heben, langsam sprechen)
        \item \dots Begrüßung
        \item \dots Vorstellung
        \item \dots Informationen über Ziele, Inhalte und Ablauf der Präsentation
    \end{itemize}

    \pause[\thebeamerpauses]
    \begin{alertblock}{Wichtig}
        Der erste Eindruck ist entscheident! Es ist hilfreich, die
        Einleitung auszuformulieren.
    \end{alertblock}
\end{frame}

\begin{frame}{So nicht}
    Zu vermeiden ist bei Redebeginn \dots
    \begin{itemize}[<+->]
        \item \dots das Beginnen mit Füllwörtern, z.B. "`Also"', "`So"', "`Ok"'
        \item \dots Hinweise darauf zu machen, dass man nicht vorbereitet ist
        \item \dots schon zu Beginn anzusprechen, dass die Zeit nicht reicht
        \item \dots auf eine zu geringe Zuhörerzahl hinzuweisen
        \item \dots die eigene Nervosität, Unsicherheit oder Befangenheit heraus zu stellen
    \end{itemize}

    \pause[\thebeamerpauses]
    \begin{block}{Grund}
        Negative Vorbemerkungen lenken die Aufmerksamkeit der
        Zuhörer erst auf diese Themen und reduzieren die Erwartungen
        des Zuhörers. Es kann zu selbsterfüllenden Prophezeihungen
        kommen.
    \end{block}
\end{frame}

\begin{frame}{Einleitungsvarianten}
    Nach Lehmmermann in Allhoff/Allhoff 1997:
    \begin{itemize}[<+->]
        \item \textbf{Das Erlangen des Wohlwollens}: Es wird mit einer
              heiteren Bemerkung, einer persönlichen Ansprache oder mit
              Bezug auf das Publikum begonnen.
        \item \textbf{Aufhänger-Technik}: Dabei wird bereits schlaglichtartig
              das zu behandelnde Thema angesprochen, z.B. durch ein
              Beispiel
        \item \textbf{Denkreiz-Technik}: Soll das Interesse der
              Zuhörer wecken oder beitet auch oft eine manipulative
              Vorinformation, z.B. durch die Bündelung von
              unkommentierten Fakten, durch unkommentierte Meinungen
              oder eine Aneinanderreihung von rhetorischen Fragen
        \item \textbf{Direkt-Technik}: Diese Technik verzichtet
              bewusst auf eine Einführung ins Thema. Sie empfiehlt
              sich bei kurzen Sachbeiträgen in Gesprächen.
    \end{itemize}
\end{frame}

\begin{frame}{Abschließende Tipps}
    \begin{itemize}[<+->]
        \item[] Der Redestart ist oft ein angespannter Augenblick
        \item[$\Rightarrow$] Such dir einen Zuhörer, der freundlich und aufmunternd aussieht
        \item[$\Rightarrow$] Nutze ihn als "`positiven Augen-Anker"'
        \item[] Wiederstehe dem Drang, mit deinem Blick an einem
                kritisch verzogenen, kopfschüttelnden Zuhörergesicht zu verharren
    \end{itemize}
\end{frame}
