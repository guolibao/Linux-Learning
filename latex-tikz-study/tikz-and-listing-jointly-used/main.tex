%%%%%%%%%%%%%%%%%%%%%%%%%%%%%%%%%%%%%%%%%%%%%%%%%%%%%%%%%%%%%%%
%
% Welcome to Overleaf --- just edit your LaTeX on the left,
% and we'll compile it for you on the right. If you open the
% 'Share' menu, you can invite other users to edit at the same
% time. See www.overleaf.com/learn for more info. Enjoy!
%
%%%%%%%%%%%%%%%%%%%%%%%%%%%%%%%%%%%%%%%%%%%%%%%%%%%%%%%%%%%%%%%
% Author: Valeria Borodin
\documentclass[border={35pt 10pt 150pt 60pt}, % left bottom right top
  svgnames]{standalone} 
%%%<
\usepackage{verbatim}
%%%>
\begin{comment}
:Title: An example of tikz and listings jointly use
:Tags: Computer science;Overlays;Arrows
:Author: Valeria Borodin
:Slug: tikz-listings

This example illustrates the joint use of
tikz and listings packages.
It represents the anatomy of a C function.
Compile twice for correct positioning. 
\end{comment}
\usepackage{tikz}
\usetikzlibrary{positioning}
\usepackage{listings}
\lstset{%
  frame            = tb,    % draw frame at top and bottom of code block
  tabsize          = 1,     % tab space width
  numbers          = left,  % display line numbers on the left
  framesep         = 3pt,   % expand outward
  framerule        = 0.4pt, % expand outward 
  commentstyle     = \color{Green},      % comment color
  keywordstyle     = \color{blue},       % keyword color
  stringstyle      = \color{DarkRed},    % string color
  backgroundcolor  = \color{WhiteSmoke}, % backgroundcolor color
  showstringspaces = false,              % do not mark spaces in strings
}
\begin{document}
\begin{lstlisting}[language = C++, numbers = none, escapechar = !,
    basicstyle = \ttfamily\bfseries, linewidth = .6\linewidth] 
 int!
   \tikz[remember picture] \node [] (a) {};
 !puissance!
   \tikz[remember picture] \node [] (b) {};
 !(int x,!
   \tikz[remember picture] \node [] (c){};
 !int n) { 

     int i, p = 1; !\tikz[remember picture] \node [] (d){};!           

     for (i = 1; i <= n; i++) 
       p = p * x; !\tikz[remember picture] \node [inner xsep = 40pt] (e){};! 

     return p; !
       \tikz[remember picture] \node [] (f){};!  
 }
\end{lstlisting}
\begin{tikzpicture}[remember picture, overlay,
    every edge/.append style = { ->, thick, >=stealth,
                                  DimGray, dashed, line width = 1pt },
    every node/.append style = { align = center, minimum height = 10pt,
                                 font = \bfseries, fill= green!20},
                  text width = 2.5cm ]
  \node [above left = .75cm and -.75 cm of a,text width = 2.2cm]
                             (A) {return value type};
  \node [right = 0.25cm of A, text width = 1.9cm]
                             (B) {function name};
  \node [right = 0.5cm of B] (C) {list of formal parameters};
  \node [right = 4.cm of d]  (D) {local variables declaration};
  \node [right = 2.cm of e]  (E) {instructions};
  \node [right = 5.cm of f]  (F) {instruction \texttt{\bfseries return}};  
  \draw (A.south) + (0, 0) coordinate(x1) edge (x1|-a.north);
  \draw (B.south) + (0, 0) coordinate(x2) edge (x2|-b.north);
  \draw (C.south) + (0, 0) coordinate(x3) edge (x3|-c.north);
  \draw (D.west) edge (d.east) ;
  \draw (E.west) edge (e.east) ;  
  \draw (F.west) edge (f.east) ;
\end{tikzpicture} 
\end{document}